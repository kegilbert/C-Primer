%%%%%%%%%%%%%%%%%%%%%%%%%%%%%%%%%%%%%%%%%%%%%%%%%%%%%%%%%%%%%%%%%%%%%%
% LaTeX Example: Project Report
%
% Source: http://www.howtotex.com
%
% Feel free to distribute this example, but please keep the referral
% to howtotex.com
% Date: March 2011 
% 
%%%%%%%%%%%%%%%%%%%%%%%%%%%%%%%%%%%%%%%%%%%%%%%%%%%%%%%%%%%%%%%%%%%%%%
% How to use writeLaTeX: 
%
% You edit the source code here on the left, and the preview on the
% right shows you the result within a few seconds.
%
% Bookmark this page and share the URL with your co-authors. They can
% edit at the same time!
%
% You can upload figures, bibliographies, custom classes and
% styles using the files menu.
%
% If you're new to LaTeX, the wikibook is a great place to start:
% http://en.wikibooks.org/wiki/LaTeX
%
%%%%%%%%%%%%%%%%%%%%%%%%%%%%%%%%%%%%%%%%%%%%%%%%%%%%%%%%%%%%%%%%%%%%%%
% Edit the title below to update the display in My Documents
%\title{Project Report}
%
%%% Preamble
\documentclass[paper=a4, fontsize=11pt]{scrartcl}
\usepackage[T1]{fontenc}
\usepackage{fourier}

\usepackage[english]{babel}			% English language/hyphenation
\usepackage[protrusion=true,expansion=true]{microtype}	
\usepackage{amsmath,amsfonts,amsthm}   % Math packages
\usepackage[pdftex]{graphicx}	
\usepackage{url}
\usepackage{lipsum}
\usepackage{listings}
\usepackage{color}
\usepackage{framed}
\usepackage{graphicx}
\usepackage{hyperref}

%%% Custom sectioning
\usepackage{sectsty}
\allsectionsfont{\centering \normalfont\scshape}
\usepackage[top=1in, bottom=0.75in, left=1in, right=1in,headsep=1cm]{geometry}

\newlength\tindent
\setlength{\tindent}{\parindent}
\setlength{\parindent}{0pt}
\renewcommand{\indent}{\hspace*{\tindent}}

%%% Custom headers/footers (fancyhdr package)
\usepackage{fancyhdr}
\pagestyle{fancyplain}
\fancyhead{}						% No page header
\fancyfoot[L]{}						% Empty 
\fancyfoot[C]{}						% Empty
\fancyfoot[C]{\thepage}				% Pagenumbering
\renewcommand{\headrulewidth}{0pt}	% Remove header underlines
\renewcommand{\footrulewidth}{0pt}		% Remove footer underlines
\setlength{\headheight}{13.6pt}


%%% Equation and float numbering
\numberwithin{equation}{section}		% Equationnumbering: section.eq#
\numberwithin{figure}{section}			% Figurenumbering: section.fig#
\numberwithin{table}{section}			% Tablenumbering: section.tab#


%%% Maketitle metadata
\newcommand{\horrule}[1]{\rule{\linewidth}{#1}} 	% Horizontal rule

\title{
	%\vspace{-1in} 	
	\usefont{OT1}{bch}{b}{n}
	\normalfont \normalsize \textsc{University of Texas at Austin\\IEEE Robotics and Automation Society} \\ [25pt]
	\horrule{0.5pt} \\[0.4cm]
	\huge Introduction to C Programming \\ 
		\vskip 0.1in
		\large Embedded System Level Design
	\horrule{2pt} \\[0.5cm]
}
\author{
	\normalfont 			\normalsize
        Kevin Gilbert\\[-3pt]		\normalsize
	21 October 2014
}
\date{}

%%% Begin document
\begin{document}
\maketitle
\section{Overview}
This document is intended to provide a crash course to embedded level specific C programming, and assumes you have no previous experience in software development. A brief overview of assembly, machine code, and instruction set architectures will be covered in the following sections to provide a basis for understanding the mechanics behind C. Dr. Patt's EE306 textbook covers C in later chapters as well if you want more additional information. This is a highly extensive subject that spans very deeply; as a result this paper won't be enough to provide you with a full background. I hope however that this is sufficient to have a baseline working understanding of the language. Subsequent courses will go much deeper into the subject, so you will be ahead of the curve if you began to gain an understanding at this stage already. 

\section{Toolsets}
The best way to learn programming is through experience.
\href{http://codepad.org}{Codepad} or \href{http://ideone.com/}{Ideone} can be used as online compilers to help learn the mechanics of C. The later sections will contain basic examples that I recommend you follow along  and play with. The last sections will cover embedded specific level programming converging with a focus on the TI LM4F launchpad in particular. Besides the online IDEs linked to above, development in a Linux environment is great for this. The GNU Compiler Collection (GCC) and the GNU Debugger (GDB) are great command line compiler and debuggers for C. Mac OS has access to these tools through the terminal as well with some additional work. Windows has much less native support for application based C development. Compilers like Eclipse and Microsoft Visual Studios work but are much bulkier. What is also often done is emulate gcc using tools such as MinGW. However, for most embedded platforms the code is compiled to the target platform's ISA and flashed onto the chip. Most producers will include IDEs to facilitate this, most notable examples being Keil or CodeComposer by TI for the launchpads, or the Arduino IDE. As stated, the largest difference between application level C development, and embedded system level C development is the target application's architecture. Application files are targeted towards the host OS. Think of generating an executable file. Embedded compilers will create output files for the embedded application specifically. For learning the mechanics of how to program in C, I will begin with more of an application approach so that the output can be more easily seen.\\

ECE students at UT Austin can get SSH access to Linux machines on campus through this \href{http://www.ece.utexas.edu/it/remote-Linux}{link}. The bottom of the page mentions Putty, a free ssh client for windows. If you want to learn some Linux, I recommend using this tool or creating a virtual machine and running a simple Linux distribution such as Ubuntu or Debian. Ubuntu is more user friendly and is typically the least jarring coming straight from a Windows environment, but comes with a lot of unnecessary components I find. Lubuntu or Xbuntu run the same core, but have a different and more lightweight Graphical User Interface (GUI). Debian is the slightly smaller and more stable version of Ubuntu. Arch Linux is absolutely bare bones and very lightweight, but has the highest learning curve. Feel free to experiment and try out different versions.

\section{Levels of Abstraction}
The first step I would like to take is to break the levels of abstraction for software to hardware development. Dr. Yale Patt is a large proponent of this, and the following figure is taken from his course notes.
\vskip 0.1in
\includegraphics[width=\textwidth]{images/levels_of_abstraction}
\vskip 0.05in
\centerline{\textbf{Figure 1. }Levels of Software Abstraction - Dr. Patt}

\newpage
Understanding all of the above layers in full detail could be a complete course on its own, so rather than try to poorly explain everything with a glancing overview I will focus on the two middle sections: the Instruction Set Architecture (ISA) and the Program layers.

\subsection{Instruction Set Architectures}
For those of you currently taking Introduction to Computing (EE306), this concept will be more familiar to you. At the lowest level, a computer provides a way to control electrons. This is done through the application of potential energy (voltage) and current in the form of circuits. These circuits in turn are organized through the microarchitecture. This is where the software and hardware begin to merge and blur the lines between the two levels. The ISA comprises the set of opcodes implemented by the microarchitecture. An example would be what most computers run today. the x86 ISA makes up most desktop and laptop cores. However how x86 is implemented in hardware is ambiguous. The Intel i3, i5, and i7 for example are all x86 processors, but implement how the instructions are executed in different manners through the microarchitecture. \\

The reason I mention this is because it is fundamental to understanding the power C has, and why C in particular is used for embedded programming. In terms of abstractions, C is very close to the hardware and ISA. The C language maps very closely to assembly. The particular set of opcodes and registers used depends on the end application. When you compile a C file with GCC on your desktop running with an x86 processor, an x86 binary file is generated. When you compile your Robotathon code in Keil, an ARM binary file is generated. What I mean by binary files in this case is a file comprised of 1's and 0's. A computer operates in binary, which is obviously very difficult for humans to read. Assembly languages were created to make programming more human readable. A line of assembly gets translated into a binary string. For example the instruction to add two registers and store the results would be \\
\centerline{\textit{ADD R2, R1, R0}} \\
in an ARM machine. The opcode for ADD had a unique binary representation, as does each register field. This means that when the computer fetches this instruction to execute it, it receives a string of 1's and 0's uniquely identifiable as the above instruction. Below is an example of a simple C file being compiler into x86 assembly. The program allocates three integer variables (more on that later), assigns the first two values of 1 and 2, adds the two numbers and stores the result into the third variable. This value is then returned. The result can be seen in the image.

\begin{framed}
	\begin{lstlisting}[language=C++,
	                   directivestyle={\color{black}}
        	            emph={int,char,double,float,unsigned},
                	     emphstyle={\color{blue}}
                	    ]
	int main(void) {
		int a, b, c;
		a = 1;
		b = 2;
		c = a + b;
		return c;
	}	
	\end{lstlisting}
\end{framed}

\newpage

\fbox{\includegraphics[width=\textwidth]{images/c_to_x86}}
\vskip 0.05in
\centerline{\textbf{Figure 2. }Simple C file compiled into x86}
\vskip 0.1in

The important portions to note in Figure 2 is the value returned by the function seen following the echo command (look for the '3') and the bottom half of the image in which the gcc generated assembly file is viewed. The GCC flag -S created the assembly file but did not generate the executable binaries. This program will not execute on its own. The -o flag following this created the executable that was subsequently run. The \textit{"ls"} commands list the files in the current directory in an Unix system. You can see the newly generated files following each GCC call.

\subsection{Programs}
One can see why C is considered such a strong low level language due to how thin the layer is between the program and ISA section. Unlike programs like Java or Python in which there are many additional layers of abstractions, C is translated directly into machine code. Starting at the other end of the levels of abstraction flow graph, we start off at the problem and algorithm stages. In terms of Robotathon this is where you will begin. Your end goal is to look at the overall problem, break the problem down into smaller modules and develop algorithms to solve them. Once you have an idea of how you can solve the problem in a step by step manner, you reach the program level. This is the level where the ambiguity of human language needs to be removed. At its core that is the purpose for programming languages. They give humans a way to translate our spoken language into a format that can control electrons all the way at the transistor level.

\section{Mechanics of C}
A C program consists of a set of sequentially executed instructions. Each line is terminated with a semi-colon to delimit the end of an instruction. These instructions will typically execute on data stored in variables. 

\subsection{Variables}
Variables are memory locations that are allocated as temporary storage space. C is very powerful in the sense that you execute very closely to the physical memory, which is why I try to emphasize some of that here. There are a variety of data types within C that you must use to define a variable. Some of the more common data types include:
\begin{itemize}
	\item char - character
	\item int - integer
	\item float - floating point
\end{itemize}

Characters variables means that a single byte of memory is allocated. Integer values depend on the architecture's word size. If your computer is a 64 bit machine, an integer variable allocates 64 bits. Likewise, a 32 bit architecture allocates 32 bits. Chars are typically used to represent ASCII encoded letters (although it is important to note that in memory it is still just a set of bits. This means depending on how you interpret the bits it can be a number as well), while integers are used to represent whole numbers. Floating point variables allow us to use decimal or fractional values. Many embedded systems do not have hardware support for floating point numbers making them very expensive to use as they require a lot of additional software math to do basic arithmetic (this is one of the nice things about the LM4F launchpads: they contain a hardware floating point unit). C, as well as many other languages, has the capability to bundle together a set of variables into a package. This package is called a \textit{structure} or struct. 
\begin{framed}
	\begin{lstlisting}[language=C++,
	                   directivestyle={\color{black}}
        	            emph={int,char,double,float,unsigned},
                	     emphstyle={\color{blue}}
                	    ]
	struct complex {
		int real;
		int imaginary;
	};
	...
	int main(void) {
		struct complex a;
		a.real = 1;
		a.imaginary = 2;
	}
	\end{lstlisting}
\end{framed}
\vskip 0.1in
\centerline{\textbf{Figure 3. }Example of a structure definition and creation}
\vskip 0.1in

A structure acts as a new data type in most cases as can be seen in Figure 3. A new variable \textit{'a'} is created of type struct complex. What this means is that in the above example, two sequential memory words will be allocated for each integer in the structure. These two variables are accessed with a "." as seen above followed by the name of the desired sub component. A nice feature in C is the ability to use \textit{"typedefs"} to relabel things. It can be a bit annoying to type out "struct complex variablename" each time. Following the structure, one can type "typedef struct complex complex". This will remap the type "struct complex" to the type "complex". The basic format of typedef is as follows: 
\centerline{\textbf{typedef} [original name] [new name mapping]} 
This line can actually be combined with the struct definition as seen below in Figure 4 by replacing the [original name] component with the actual struct definition.\\

\fbox{\includegraphics[width=\textwidth]{images/struct_sample}}
\vskip 0.05in
\centerline{\textbf{Figure 4. }Simple example of structures in C}
\vskip 0.1in

There is a lot of extra stuff in the image that we haven't really gotten to yet so I apologize for that. I will try my best to provide a step by step explanation of a sample file once I've covered the different structures within C.
\subsection{Flow of Execution}
While arithmetic and assignment operations are vital, we need to be able to actually control the flow of our program. This is done with three primary structures:
\begin{itemize}
	\item If-Statement - If a condition holds true, execute the following code segment. Can be followed by an \textit{else} block to create an if-else statement.
	\item While-Loop - While a condition holds true, execute the following code segment.
	\item For-Loop - Typically used for counting, execute the following code segment until the condition is no longer valid.
\end{itemize}

\newpage

\begin{framed}
	\begin{lstlisting}[language=C++,
	                   directivestyle={\color{black}}
        	            emph={int,char,double,float,unsigned},
                	     emphstyle={\color{blue}}
                	    ]
	while(x==1) {
		// do things while x is 1
	}
	...
	for(i=0; i<10; i=i+1) {
		// do things while i is less than 10.
		// increment i each iteration
	}
	...
	if(x==1) {
		// Execute this segment if x is 1.
	} else if(x==2) {
		// Execute this if x is 2 AND x is not 1.
	} else {
		// Execute this if x is not 1 or 2.
	}
	\end{lstlisting}
\end{framed}
\vskip 0.1in
\centerline{\textbf{Figure 5. }Example of a structure definition and creation}
\vskip 0.1in

You'll notice the logic syntax I used for checking to see if a variable was equal to a particular value was of the form "x == 1" rather than "x = 1". A single equal sign acts as an assignment operator, meaning that in the case of "x = 1" you are assigning the value of decimal 1 to the variable x. Booleans (true,false) are not natively supported in C. Instead 0 and 1 are used to represent true and false. This means that the operation "1 == 1" for example with return "1", meaning that the two values are equal. Likewise "5 < 2" will return a 0 as 5 is greater than two. There are quite a few online resources that have a more comprehensive listing of C logic and arithmetic operators that I recommend you skim through later to just get an idea of how to manipulate data.

\subsection{Functions}
You'll notice in all of my examples I have the same line: \textit{"int main(void)"}. This is an example of a function. In a microcontroller, there exists a block of reserved memory called a "vector table". They contain memory locations of important system functions. The reset vector for example contains the address of the first instruction of the user program flashed onto the board. Whenever the system is reset, the program counter (PC) is loaded with the value stored in the reset vector and execution begins. This starting address is found by the "main" function. Your program begins within the main function. A function in C follows the basic structure of: \\
\centerline{returnValue \textbf{functionName}(param1, param2, param3,...)}\\
So in the example of main, the function returns an integer, and takes in no parameters. Everything enclosed within the '{ }' are part of the function. The calling convention for a function is:\\
\centerline{var return = \textbf{function}(param1, param2);}\\
When creating a function, you typically include a function prototype or definition to allow the compiler to recognize the method. You will often see these included in header files to allow a file to use functions from another location. Look at RASWare2013 as an example of this. Below are two basic examples of a function being made and used. The first image is a completely explicit example, whereas the second image has the exact same functionality, but a more condensed codebase by using implicit data flow.

\fbox{\includegraphics[width=\textwidth]{images/basic_function}}
\vskip 0.05in
\centerline{\textbf{Figure 6. }Simple example of a function that determines whether or not a number is even}
\vskip 0.1in

\fbox{\includegraphics[width=\textwidth]{images/advanced_function}}
\vskip 0.05in
\centerline{\textbf{Figure 7. }The same function as in figure 6, but more condensed}
\vskip 0.1in

\subsubsection{Input/Output (I/O)}
You'll see that in most of my examples I have used a function called printf. This function displays data as a string onto the output console. The "\%d" portion within the string parameter will take in a decimal (integer) value and print it as a string. There are several different formats you can use. For example, "\%x" displays the integer value in hexadecimal. "\%0.8x" will display a hexadecimal value with up to 8 preceding 0's. This is helpful to column align values with unmatched lengths (a 64 bit value that contains 0x12 would be printed as 0x00000012). The value that gets supplied comes from printfs following input parameters. So printf("hello \%d", 10): would print "hello 10". The \\n seen in most of my printf statements is a newline. Looking far back at Figure 4 I added an example of printing a character as both a char and as the hexadecimal value.

\fbox{\includegraphics[width=\textwidth]{images/simple_printf}}
\vskip 0.05in
\centerline{\textbf{Figure 8. }Simple printf example}
\vskip 0.1in
In Figure 8 above you can see that although I assigned a value of 300 to the variable a, 44 was printed out as the value. This is because again a char allocates a byte, or 8 bits, or data. 300 requires more than 8 bits to represent, so any bits after bit 8 are truncated, or dropped. \\

A quick additional side note is the \#include that is at the top of most of my examples. Pre-processor statements are indicated by "\#"s in C. These are notes to the compiler and are not compiled. \#include links to a header file that provides additional functions. In this case stdio.h includes I/O functions such as printf. 

\subsection{Pointers}
Pointers are a very powerful tool. As a mechanic they are very simplistic. Essentially pointers are variables that contain a memory address that points to another variable. The * and \& operators are typically used to create and dereference pointers. The * dereferences while the \& means to read the memory address a variable is stored at. Dereferencing a pointer means that you are reading the data stored at the memory location in your pointer variable.

\fbox{\includegraphics[width=\textwidth]{images/pointer}}
\vskip 0.05in
\centerline{\textbf{Figure 9. }Pointer memory layout diagram}
\vskip 0.1in

The printfs in Figure 9 would return "0x3008" and 'a' respectively.
\section{Embedded Development and RASWare} 
Because we are developing on an embedded system, we get access to GPIO pins to allow us to communicate with the outside world. The goal of RASware is to simplify and create an abstraction layer for initialization and calling routines. I recommend going through the RASDemo files as well as the RASLib include files to look at the function prototypes. I apologize for the cramped and rushed nature of this paper and realize it may still not help much. I will be around the office and can go over in person some more of the subject.

%%% End document
\end{document}
